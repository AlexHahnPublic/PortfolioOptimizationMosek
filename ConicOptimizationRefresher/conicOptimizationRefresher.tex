\documentclass{article}

\usepackage[english]{babel}
\usepackage[utf8]{inputenc}
\usepackage{amsmath,amssymb}
\usepackage{parskip}
\usepackage{graphicx}

% Margins
\usepackage[top=2.5cm, left=3cm, right=3cm, bottom=4.0cm]{geometry}
% Colour table cells
\usepackage[table]{xcolor}

% Get larger line spacing in table
\newcommand{\tablespace}{\\[1.25mm]}
\newcommand\Tstrut{\rule{0pt}{2.6ex}}         % = `top' strut
\newcommand\tstrut{\rule{0pt}{2.0ex}}         % = `top' strut
\newcommand\Bstrut{\rule[-0.9ex]{0pt}{0pt}}   % = `bottom' strut

%%%%%%%%%%%%%%%%%
%     Title     %
%%%%%%%%%%%%%%%%%
\title{Conic Optimization Refresher}
\author{Alex Hahn \\ from mosek cookbook}
\date{\today}

\begin{document}
\maketitle

%%%%%%%%%%%%%%%%%
%   intro formulation   %
%%%%%%%%%%%%%%%%%
\section{Types of Cones}
Conic optimization is a class of convex optimization problems. The geneneral form of a conic optimization problem is:

$$\text{maximize } \mathbf{c}^T\mathbf{x}$$
$$\text{subject to } \mathbf{Ax} + \mathbf{b} \in \mathcal{K} $$
where $\mathcal{K}$ is a product of the following basic types of cones:

\begin{itemize}
\item{\textbf{Linear cone:}}
\end{itemize}
$$\mathbb{R}, \mathbb{R}^n_+, {0}$$

\begin{itemize}
\item{\textbf{Quadratic cone and rotated quadratic cone:}}
\end{itemize}
The quadratic cone is the set
$$\mathcal{Q}^n = \left\{ x \in \mathbb{R} \bigg| x_1 \geq \sqrt{x_2^2 +\cdots+ x_n^2}\right\}$$
The rotated quadratic cone is the set
$$\mathcal{Q}_r^n = \left\{ x \in \mathbb{R} \bigg|2x_1x_2\geq x_3^2+\cdots + x_n^2, x_1, x_2\geq 0\right\}$$

Together the union of these two cones covers the class of SOCO (second-order cone optimization)
problems which includes all QO (quadratic optimization) and QCQO (quadratically constrained
quadratic optimization) problems as well.

\begin{itemize}
\item{\textbf{Primal power cone:}}
\end{itemize}

$$\mathcal{P}^{\alpha, 1-\alpha}_n = \left\{ x \in \mathcal{R}^n\bigg|x_1^\alpha x_2^{1-\alpha}\geq\sqrt{x_3^2+\cdots+x_n^2},x_1,x_2 \geq 0\right\}$$

\begin{itemize}
\item{\textbf{Primal exponential cone:}}
\end{itemize}

$$K_{\text{exp}} = \left\{ x \in \mathcal{R}^3\bigg|x_1\geq x_2\text{exp}\left(\frac{x_3}{x_2}\right),x_1,x_2 \geq 0\right\}$$

\begin{itemize}
\item{\textbf{Semidefinite cone:}}
\end{itemize}

$$\mathcal{S}^n_+=\{X \in \mathbb{R}^{n\times n}| X\text{ is symmetric positive semidefinite}$$

Semidefinite cones model SDO problems.

Each of these cones allow formulating different types of convex constraints.

\section{Selection of Conic Constraints}

Examples of real world constraints (financial)and how to convert them to conic form.

\subsection{Maximum function}

Model the maximum constraint max($x_1, x_2, ... , x_n)\leq c$ using n linear constraints introduces with an auxiliary variable $t$:

\begin{align}
  \begin{split}
    t &\leq c, \\
    t &\geq x_1, \\
    &\vdots \\
    t &\geq x_n.
  \end{split}
\end{align}

For example we can write the constraints max$(x_i,0)\leq c_i, 1,...,n$ as
$$\mathbf{t} \leq \mathbf{c}, \mathbf{t}\geq \mathbf{x}, \mathbf{t}\geq \mathbf{0},$$
where \textbf{t} is an $n$-dimensional vector.

\subsection{Positive and negative part}

A special case of modeling the maximum function is to model the positive part $x^+$ and the negative part $x^-$ of a variable $x$.
We define these as $x^+=\text{max}(x,0)$ and $x^-=\text{max}(-x,0)$. Model them explicitly with the above methodology and
the inequatlities $x^+=\text{max}(x,0)$ and $x^-=\text{max}(-x,0)$, or implicitly with the constraints:


$$x = x^+ - x^-,$$
$$x^+, x^- \geq 0.$$

Note, there is still a degree of freedom in both the implicit and explicit magnitudes of $x^+$ and $x^-$. In the
explicit case we have inequalities and in the implicit case only the difference of the variables is constrained.
Ultimately it is possible for both $x^+$ and $x^-$ to be positive, allowing optimal solutions where $x^+ = \text{max}(x,0)$
and $x^-1=\text{max}(-x,0)$ does not hold.

Theoretically we can introduce a complimentarity contraint $x^+x^-=0$ (or $\langle\textbf{x}^+,\textbf{x}^-\rangle=0$)
but that is non-convex/ can't be modeled. There are two workarounds to ensure that $x^+ = \text{max}(x,0)$
and $x^-1=\text{max}(-x,0)$ hold the optimal solution: 1) penalize the magnitude of the two solutions, so if both
are positive in any one solution, the solver could always improve the objective by reducing them until either one
becomes zero. 2) formulate a mixed integer problem.

\subsection{Absolute value}
We can model the absolute value constraint $|x| \leq c$ by using the maximum function observing that
$|x|=\text{max}(x,-x)$:
$$-c\leq x\leq c$$

Or you could use a quadratic cone:
$$(c,x)\in \mathcal{Q}^2$$

\subsection{Sum of largest elements}

The sum of the m largest elements of a vector $\mathbf{x}$ is the optimal solution of the LO problem:
$$\text{maximize } \mathbf{x}^T\mathbf{z}$$
$$\text{subject to } \mathbf{1}^T\mathbf{z}=m,$$
$$\mathbf{0}\leq\mathbf{z}\leq\mathbf{1}$$

Here \textbf{x} cannot be a variable as this would result in a nonlinear objective. Looking at the dual of
the problem:
$$\text{minimize } mt + \mathbf{1}^T\mathbf{u}$$
$$\text{subject to } \mathbf{u} + t \geq \mathbf{x},$$
$$\mathbf{u} \geq 0.$$

This is the same problem as min$_t$ $mt + \sum_i\text{max}(0,x_i-t)$, in which $x$ can be a variable and thus optimized.

\subsection{Linear combination of largest elements}

Selecting \textbf{z} tot have an upper bound $\mathbf{c}\geq\mathbf{0}$, and a real number
$0\leq b \leq c_\text{sum}$ instead of integer $m$, where $c_\text{sum}=\sum_ic_i$:
$$\text{maxmize } \mathbf{x}^T\mathbf{z}$$
$$\text{subject to } \mathbf{1}^T\mathbf{Z} = c_\text{sum} - b,$$
$$\mathbf{0}\leq\mathbf{z}\leq\mathbf{c}.$$

This has the optimal objective $c^\text{frac}x_{ib} + \sum_{i>i_b}c_ix_i$ where $i_b$ is such that
$\sum_{i=1}^{i_b-1}c_i<b\leq\sum^{i_b}_{i=1}c_i$, and $c_{i_b}^\text{frac}=\sum^{i_b}_{i=1}c_i-b<c_{i_b}.$
The dual of this problem:

$$\text{minimize } (c_\text{sum}-b)t+\mathbf{c}^T\mathbf{u}$$
$$\text{subject to } \mathbf{u}+t\geq\mathbf{x},$$
$$\mathbf{u}\geq\mathbf{0}.$$

which is the same as $\text{min}_t(c_\text{sum}-b)t+\sum_ic_i\text{max}(0,x_i-t)$.












\end{document}
