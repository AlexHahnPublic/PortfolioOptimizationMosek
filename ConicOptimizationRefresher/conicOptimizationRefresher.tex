\documentclass{article}

\usepackage[english]{babel}
\usepackage[utf8]{inputenc}
\usepackage{amsmath,amssymb}
\usepackage{parskip}
\usepackage{graphicx}

% Margins
\usepackage[top=2.5cm, left=3cm, right=3cm, bottom=4.0cm]{geometry}
% Colour table cells
\usepackage[table]{xcolor}

% Get larger line spacing in table
\newcommand{\tablespace}{\\[1.25mm]}
\newcommand\Tstrut{\rule{0pt}{2.6ex}}         % = `top' strut
\newcommand\tstrut{\rule{0pt}{2.0ex}}         % = `top' strut
\newcommand\Bstrut{\rule[-0.9ex]{0pt}{0pt}}   % = `bottom' strut

%%%%%%%%%%%%%%%%%
%     Title     %
%%%%%%%%%%%%%%%%%
\title{Conic Optimization Refresher}
\author{Alex Hahn \\ from mosek cookbook}
\date{\today}

\begin{document}
\maketitle

%%%%%%%%%%%%%%%%%
%   intro formulation   %
%%%%%%%%%%%%%%%%%
\section{Types of Cones}
Conic optimization is a class of convex optimization problems. The geneneral form of a conic optimization problem is:

$$\text{maximize } \mathbf{c}^T\mathbf{x}$$
$$\text{subject to } \mathbf{Ax} + \mathbf{b} \in \mathcal{K} $$
where $\mathcal{K}$ is a product of the following basic types of cones:

\begin{itemize}
\item{\textbf{Linear cone:}}
\end{itemize}
$$\mathbb{R}, \mathbb{R}^n_+, {0}$$

\begin{itemize}
\item{\textbf{Quadratic cone and rotated quadratic cone:}}
\end{itemize}
The quadratic cone is the set
$$\mathcal{Q}^n = \left\{ x \in \mathbb{R} \bigg| x_1 \geq \sqrt{x_2^2 +\cdots+ x_n^2}\right\}$$
The rotated quadratic cone is the set
$$\mathcal{Q}_r^n = \left\{ x \in \mathbb{R} \bigg|2x_1x_2\geq x_3^2+\cdots + x_n^2, x_1, x_2\geq 0\right\}$$

Together the union of these two cones covers the class of SOCO (second-order cone optimization)
problems which includes all QO (quadratic optimization) and QCQO (quadratically constrained
quadratic optimization) problems as well.

\begin{itemize}
\item{\textbf{Primal power cone:}}
\end{itemize}

$$\mathcal{P}^{\alpha, 1-\alpha}_n = \left\{ x \in \mathcal{R}^n\bigg|x_1^\alpha x_2^{1-\alpha}\geq\sqrt{x_3^2+\cdots+x_n^2},x_1,x_2 \geq 0\right\}$$

\begin{itemize}
\item{\textbf{Primal exponential cone:}}
\end{itemize}

$$K_{\text{exp}} = \left\{ x \in \mathcal{R}^3\bigg|x_1\geq x_2\text{exp}\left(\frac{x_3}{x_2}\right),x_1,x_2 \geq 0\right\}$$

\begin{itemize}
\item{\textbf{Semidefinite cone:}}
\end{itemize}

$$\mathcal{S}^n_+=\{X \in \mathbb{R}^{n\times n}| X\text{ is symmetric positive semidefinite}$$

Semidefinite cones model SDO problems.

Each of these cones allow formulating different types of convex constraints.

\section{Selection of Conic Constraints}

Examples of real world constraints (financial)and how to convert them to conic form.

\subsection{Maximum function}

Model the maximum constraint max($x_1, x_2, ... , x_n)\leq c$ using n linear constraints introduces with an auxiliary variable $t$:

\begin{align}
  \begin{split}
    t &\leq c, \\
    t &\geq x_1, \\
    &\vdots \\
    t &\geq x_n.
  \end{split}
\end{align}

For example we can write the constraints max$(x_i,0)\leq c_i, 1,...,n$ as
$$\mathbf{t} \leq \mathbf{c}, \mathbf{t}\geq \mathbf{x}, \mathbf{t}\geq \mathbf{0},$$
where \textbf{t} is an $n$-dimensional vector.

\textbf{Positive and negative part}//

A special case of modeling the maximum function is to model the positive part $x^+$ and the negative part $x^-$ of a variable $x$.
We define these as $x^+=\text{max}(x,0)$ and $x^-=\text{max}(-x,0)$. Model them explicitly with the above methodology and
the inequatlities $x^+=\text{max}(x,0)$ and $x^-=\text{max}(-x,0)$, or implicitly with the constraints:

\begin{align}
  x &= x^+ - x^-, \\
  x^+, x^- &\geq 0.
\end{align}

\end{document}
